\chapter{Evaluation}

%\begin{itemize}
%   \item Were the requirements correctly identified? 
%   \item Were the design decisions correct?
%   \item Could a more suitable set of tools have been chosen?
 %  \item How well did the software meet the needs of those who were expecting to use it?
%   \item How well were any other project aims achieved?
%   \item If you were starting again, what would you do differently?
%\end{itemize}

%Such material is regarded as an important part of the dissertation; it should demonstrate that you are capable not only of carrying out a piece of work but also of thinking critically about how you did it and how you might have done it better. This is seen as an important part of an honours degree. 

%There will be good things and room for improvement with any project. As you write this section, identify and discuss the parts of the work that went well and also consider ways in which the work could be improved. 

%Review the discussion on the Evaluation section from the lectures. A recording is available on Blackboard. 


\section{Requirements}

In terms of meeting the requirements as detailed in Section~\ref{req} the system is entirely successful. Each requirement is implemented in a fairly robust and well tested manner. In terms of the initial identification of gathering of requirements there was room for improvement. Essentially the requirements for the system were fairly straight forward to accomplish and did not provide a large degree of challenge or the opportunity for significantly interesting work. The reason for this was essentially a reaction to the change in project topic, these final requirements were fairly rushed into agreement and should have been discussed in more exhaustive detail with the stakeholders involved. However they were sufficient in defining a system that solved the problem objective.




\section{Implemented System}
\subsection{Strengths}
The delivered system is well tested and developed in a robust way with a focus on maintainability. The choice of framework and system architecture provide an environment which is scalable, expandable and highly portable.

The system allows users to easily browse through collections of plants and associated data for given experiments. Navigation through, and interaction with the system is simple and intuitive. Users are able to create graphs of data in a flexible way and are not limited by system imposed restrictions on the data due to implementation choices such as object typing and data structures.

A convention over configuration approach makes managing and administering the system simple and efficient whilst also providing the emans for more detailed manipulation of system parameters if required. 

\subsection{Weaknesses}





\subsection{Future Work and Improvements}

The project brief was fairly open ended and the delivered system fairly narrow in scope, this leaves many avenues for potential improvement and expansion of the system, potential examples of which are discussed  below :

\subsubsection{Search and filtering}
Further utility could be provided by the system if some extensive search and filtering functionality was implemented. Filtering the time series of images or plants in an experiment by attribute values, ranges etc would enhance the ability of the user to find specific information of interest or aid in inputting such specific data to particular entities within the system. Similarly, searching for plants or attribute values would be a helpful utility in both exploration of the experiment data and input of data to specific entities of interest within the system.

One approach which was partially investigated during the implementation was to use the Elasticsearch\cite{elastic} integration provided by Spring. This implementation provides a means to index certain attributes and classes within the system to provide a convenient and fast solution to searching and filtering. The approach follows a similar principle to that implemented for the persistence layer, that is, the entity classes are annotated and a `data repository' interface is used to query these entities, further details can be found in the Spring documentation\cite{_spelastic}  


\subsubsection{Comparisons and statistical analysis}
As the interesting results of experiments carried out at the NPPC are often statistical in nature it would be convenient to the scientists using the system to be able to conduct or view the results of simple statistical analysis alongside the visualisation of data provided by the delivered system. As discussed in previous sections, the scientists conducting experiments often analyse only very narrow ranges of data correlations that could be present in an experiment. Providing a correlation measure for graphed attributes could indicate to users of the system that interesting correlations exist in previously overlooked data. 

When generating a graph visualisation from two attributes, a correlation measurement could be taken between the two attributes using the entire set of examples within the experiment. One such method would be to use Spearmans rank correlation coefficient\cite{spearman} to display the correlations between attributes. There are various libraries available within the Java ecosystem that provide implementations of such statistical analysis algorithms and methods. 

\subsubsection{Machine learning}
With the potential for large datasets of meaningful, well understood data that is often statistical in nature it is an ideal scenario to leverage some machine learning capability in order to produce potentially interesting correlations and conclusions from system data. The main obstacle to providing such a facility is the current lack of collected data within the context of experiments. If the system was used mroe extensively then the provision of machine learning would become more and more feasible. Data for ongoing experiments recorded within the system could be used to provide the basis of predictions of certain traits and attributes which could, if sufficiently accurate, provide the scientists with the means to focus effort on the most interesting, promising or relevant subset of plants within an experiment.



\subsubsection{Image processing and computer vision}
The NPPC captures vast quantities of image data for each experiment. Different modalities and angles add to the richness of the images as a potential data source in themselves. There is a wide array of potential improvements to the system that could be made to take advantage of such data. The NPPC themselves are concerned with the leverage of computer vision approaches to image analysis with ongoing research into useful automated procedures such as the detection of specific growth stages in oats as developed by Boyle et al\cite{boyle_image-based_2015}. As a result there are automated processes that provide support to the deployment of image based processes such as segmentation of images to provide masks of plant pixel location within images.

The system could be extended to use popular image processing libraries such as ImageJ (integration of which was tested briefly during spikework and verified as viable) which can provide many useful functions such as pixel colour maps, counts and so forth which could be used to infer the health or age of a given plant, detect fungus and so on. 

It is possible to invoke external programs using system calls from within the delivered system, it is not infeasible to run compatible experiments through detectors such as that developed by Boyle et al\cite{boyle_image-based_2015} in order to automate the detection of certain growth stages and record these results within the persistence layer of the system


\subsubsection{Comprehensive user management}
Within the delivered system there is only the concept of an admin user and generic users. Providing the means for users to register and create account opens up many interesting options for further development of the system. In providing individual user accounts it is possible to then extend the attribute system to allow a concept of ownership, for example if a user adds attributes to a plant then that user has ownership of that attribute and gains the ability to delete or edit the attribute. Other users could raise disputes against the attribute or provide a competing value for the same attribute, this would allow the system to score agreement on particular assessments of certain traits or characteristics as identified by the different stakeholders within an experiment. Providing an agreement score for attributes in such a manner would provide a more robust dataset for use in other applications such as machine learning or vision systems.

A popular approach to encouraging user participation within such systems is gamification (bringing elements of game-design into non-game contexts), a system where users can gain points or extended privileges based on participation could be used to encourage scientists to add data and utilise the system more fully. 




\section{Process}

It is believed that an agile approach as chosen for this project was the correct choice and that the process followed during the course of this project was successful for a number of reasons. It provided an environment in which work could be effectively prioritised based on emergent or short term needs and goals without having to reorganise more than the current day or iterations work. Being able to react to change and bugs in this way made development more comfortable especially when requirements were not concrete and the domain was not fully understood at the beginning of iterations. Other approaches may not offer such a degree of flexibility.

The use of user stories and the ability to move them into and out of the scope of iterations along with progressing their status made the project work flow more easily and allowed the focus to be on the goal of an iteration and the next release rather than a generic list of un-organised tasks.  As discussed in Section~\ref{proc}, the estimation of tasks in terms of time could have been improved to provide a more complete understanding of developer throughput which would have led to more efficient allocation of pomodoros to the correct tasks. However, estimation is notoriously difficult to do with a high degree of accuracy especially when working with technologies that developers aren't fully experienced in using.

A key part of the process followed is the incremental and frequent release of the product. Providing a deployment at the end of each iteration offered a source of reassurance to stakeholders that work was progressing and also provided, via the use of version control facilities, the opportunity to rollback if a release was broken or problematic.

The use of a blog to keep up-to-date with project work was a great benefit, especially during the writing of this report where it served as a reminder of specific implementation or design details. Having to write a blog post acted as a source of accountability, encouraging the hour or so of extra work it would take to make a worthy post.

\section{Student Evaluation}
%\subsection{General}

There are a number of areas in which personal improvement has been made during the course of delivering this project and a number of skills have been gained or enhanced as a direct result of the work undertaken. These will be discussed in this section as well as a general evaluation of performance and approach during the course of the work.

In terms of preparation for the project there are two facets worthy of discussion. Firstly the provision of supporting applications and technical environment in which to conduct the work was well chosen and provided a great deal of utility and efficiency when designing, developing and testing components of the project. Use of testing frameworks and continuous integration provided a continuous assurance that errors would be found during the development process which provided more time to focus on design and implementation rather than manual testing and verification. Secondly, the attitude towards resolving unknowns in the discovery phase of the project could have affected the general results in a fairly negative way, a key lesson was learned as a result and that is a more pro-active approach early in a project is almost always going to be preferable and cheaper in terms of effort in the long run.

In evaluating potential technologies and designing the system, experience has been gained in a wide range of useful applications and frameworks. Experience gained in Spring and supporting technologies is of direct relevance to post-graduation employment and many of the skills acquired are desirable in many industries. Evidence of automated testing and appreciation of stress testing frameworks are in-demand and widely transferable skills within the software industry.

The project saw opportunities to practice many skills and apply knowledge that has been gained during the course of previous university study, notably project methodology choices and good software engineering practises such as a focus on developing for maintainability and scalability. The experienced gained in development using the Java and Javascript languages during my time in university provided a solid foundation of understanding that allowed development time to focus on new technologies such as framework specific implementations as opposed to the core  technologies.

The nature of the project and its concurrency with certain external events, along with the nature of the student, has provided a unique opportunity to appreciate stress management in ways that have not been required or considered previously.

If the project were to be repeated the biggest difference would be the integration with the staff at the NPPC, being more proactive in this regard would have ensured that all the involved stakeholders would have a better understanding of the project and the expected output. Implementing the pomodoro technique from the beginning would have resulted in higher productivity. It would have been a much more interesting project had the focus been on images as data rather than something to display. In general the approach to development and design would remain consistent but differences in personal attitude and outlook would have a significant effect on execution of the project.

 A key and unexpected take away from this project is what I learned about myself personally, rather than in any professional or academic sense and I am grateful for the opportunity to learn this lesson.
%There are a number of areas which could have been improved when considering personal performance during the course of this project. Most notably a more proactive approach to engaging with the plant scientists at the NPPC would have brought benefits to the project in terms of delivering a more useful system and focusing on providing functionality that would improve their interactions with the system. A tenuous offer of co-location was made such that a hot desk could have been available at the NPPC for the duration of project work which would have allowed a closer working relationship with the staff at the NPPC. In taking these more proactive approaches, more time could have been freed to implement some of the more interesting improvements discussed in the previous section, this is the source of a little regret and results in a changed attitude towards how work is approached in general.

%The approach to time management was greatly improved as a result of this project, when it seemed that much time was being wasted in unfocused effort, this was rectified by implementation of, and dedication to, the pomodoro technique resulting in more productive output in fewer hours then before the adoption of the technique.

%The project provided the opportunity to learn about an interesting domain that otherwise would not have been available and to meet with inspiring people who enjoy their field of work.
%\subsection{Take homes}