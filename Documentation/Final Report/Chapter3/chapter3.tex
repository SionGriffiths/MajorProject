\chapter{Implementation}



\section{Integration with NPPC data repository}
Fairly straight forward, recursive methods slow, iterative and tightly coupled fast
The structure of the data repository changed soon after implementation was begun
\section{Graphing System}
Wrapping things into nice JSON format for plotly, fun with join queries.

\section{Domain model implementation and ORM}
Talk about looping references and stack overflows? Laze/eager fetch? 

\section{Data Import}
Use of the arrays to hold column values "hashing" the values into the right place using the numbers in the list

\section{IBERS hosted environment}



As discussed in section~\ref{framework}, a consideration during the selection of framework within which the project would be implemented was the number of dependencies it would place on the environment it was hosted. The less dependencies then the more attractive the framework since this directly affects the time required to configure and maintain the environment. This project required that three dependencies be present on a given environment to enable the system to function, the Java runtime environemnt, MySQL database server and the Apache Maven build tool which is bundled into and required to run Spring-Boot applications. 

Initially the Java version used for the project was version 8, however, it was more convenient for the NPPC to provide a server with Java version 7 forcing the change in targeted language level.

The provided hosting environment was provisioned and set up by Dr Colin Sauze, the data manager at the NPPC. The hosting environemnt is a virtual machine running Ubuntu v14.04 hosted on a Intel CPU based server with 8GB of RAM. It is hosted with the Universities firewall and as such is only accessible from within the University network (or via VPN). The network restriction meant that the deployment of a release build of the system could not be completed via the project continuous integration platform. 

The deployment process was not fully automated for the project. For a given release version, the source code would be checked out from the version control repository and the system rebuilt and restarted on the server itself. Sine this required only three commands at the end of each week to be entered into the system terminal, it was deemed that automation was not necessary.
